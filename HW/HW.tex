\documentclass[12pt, a4paper]{extarticle}


% Russian text support
\usepackage[T2A]{fontenc}
\usepackage[utf8]{inputenc}
\usepackage[russian]{babel}

% Some useful packages
\usepackage{indentfirst}
\usepackage{etoolbox}
\usepackage{amsmath}
\usepackage{amsfonts}
\usepackage{xcolor}

% Page geometry
\usepackage[
    left=3cm,
    right=1cm,
    top=2cm,
    bottom=2cm
]{geometry}

% Make titles not to have numbering
\newenvironment*{dummyenv}{}{}

\newcommand{\mysection}[1]{
    \addcontentsline{toc}{section}{#1}
    \begin{dummyenv}
        \bfseries\large #1
    \end{dummyenv}
}

\makeatletter
\patchcmd{\l@section}
  {\hfil}
  {\leaders\hbox{\normalfont$\m@th\mkern \@dotsep mu\hbox{.}\mkern \@dotsep mu$}\hfill}
  {}{}
\makeatother

% Here we go...
\title{БДЗ по прикладной криптографии}
\author{Фирсов Георгий, М21-507}

\begin{document}

\maketitle

\tableofcontents

\pagebreak

\mysection{Задание 1}

Анна генерирует два числа $x \xleftarrow{R} \mathbb{Z}_1, y \xleftarrow{R} \mathbb{Z}_q$, после чего отсылает
Борису тройку $(A_0, A_1, A_2) = (g^x, g^y, g^{xy + a})$.

Борис генерирует свои два числа $r \xleftarrow{R} \mathbb{Z}_q, s \xleftarrow{R} \mathbb{Z}_q$, а затем отправляет
Анне следующую пару: $(B_1, B_2) = (A_1^r \cdot g^s, (A_2/g^b)^r \cdot A_0^s)$. Заметим, что:
\begin{equation*}
    \begin{split}
        & B_1 = A_1^r \cdot g^s = g^y \cdot g^s = g^{\textcolor{red}{y + s}} \\
        & B_2 = (A_2/g^b)^r \cdot A_0^s) = g^{xy + a} \cdot g^{-b} \cdot g^{xs} = g^{x(\textcolor{red}{y + s}) + a - b}
    \end{split}  
\end{equation*}

Если $B_1$ возвести в степень $x$ и затем умножить на обратный к полученному элемент число $B_2$, то получится $g^{a - b}$:
\begin{equation*}
    \begin{split}
        & B_1 ^ x = \left(g^{y + s}\right) ^ x = g^{x(y + s)} \\
        & B_2 \cdot (B_1^{-x}) = g^{x(y + s) + a - b} \cdot g^{-x(y + s)} = g^{a - b}
    \end{split}
\end{equation*}

Если $a = b$, то $g^{a - b} = g^0 = e_{\mathbb{G}}$. Это свойство и можно использовать для проверки равенства чисел $a$ и $b$.

\textbf{Ответ:} в) Анна проверяет равенство $B_2 / B_1^x = 1$.
\\

\mysection{Задание 2}

Так как числа $p, a, b$ общеизвестны, то считаю, что при разработке программы все \textit{возможные} вычисления с данными 
параметрами выполняются заранее (то есть, собственно, на этапе разработки программы). Несложно увидеть, что:
\begin{equation*}
    \begin{split}
        H^{(n)}(x) & = \underbrace{H_{p,a,b}(H_{p,a,b}(\cdots H_{p,a,b}(x) \cdots))}_{\text{n раз}} = 
            \underbrace{a(a(\cdots\ ax + b\ \cdots) + b) + b}_{\text{n раз}} = \\
        & = a^n x + b \sum_{j=0}^{n - 1} a^j \mod p
    \end{split}
\end{equation*}

Обозначим:
\begin{equation*}
    \begin{split}
        & a' := a^n \mod p \\
        & b' := b \sum_{j=0}^{n - 1} a^j \mod p
    \end{split}
\end{equation*}

Тогда:
\begin{equation*}
    H^{(n)}(x) = a'x + b' \mod p
\end{equation*}

Значения $a'$ и $b'$ возможно вычислить предварительно на этапе разработки программы, что позволит вычислять функцию
$H^{(n)}$ так же быстро, как и $H_{p,a,b}$.

Но может случиться так, что числа $p,a,b$ \textit{заранее} не известны (например, меняются с течением времени). 
Таким образом, возникает потребность поддержки вычисления $a', b'$ на лету. В таком случае заметим, что:
\begin{equation*}
    \begin{split}
        \sum_{j=1}^{n-1}a^j = (a^{n-1} - 1) \cdot (a - 1)^{-1} \mod p
    \end{split}
\end{equation*}

При больших $n$ возведение в степень $n-1$ потребует примерно столько же операций, сколько и возведение в степень $n$.
Заметим, что возведение в степень можно производить, пользуясь следующей идеей: $a^4 = a^2 \cdot a^2, 
a^8 = a^4 \cdot a^4$ и т.д. Данный алгоритм требует асимптотически $\log _2(n)$ умножений.

\textbf{Ответ:} а) $H^{(n)}$ может быть вычислена так же быстро, как $H_{p,a,b}$ (в случае известных заранее значений 
$p,a,b$); г) вычисление $H^{(n)}$ требует времени $O(\log n)$ (в случае неизвестных заранее значений $p,a,b$).
\\

\mysection{Задание 3}

\mysection{Задание 4}

\mysection{Задание 5}

\mysection{Задание 6}

\mysection{Задание 7}

\mysection{Задание 8}

\mysection{Задание 9}

\mysection{Задание 10}

\mysection{Задание 11}

\mysection{Задание 12}

\end{document}
