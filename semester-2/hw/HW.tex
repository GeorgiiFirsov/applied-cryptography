\documentclass[12pt, a4paper]{extarticle}


% Russian text support
\usepackage[T2A]{fontenc}
\usepackage[utf8]{inputenc}
\usepackage[russian]{babel}

% Some useful packages
\usepackage{indentfirst}
\usepackage{etoolbox}
\usepackage{amsmath}
\usepackage{amssymb}
\usepackage{amsfonts}
\usepackage{xcolor}

% Pictures support
\usepackage{graphicx}
\graphicspath{ {./images/} }

% Page geometry
\usepackage[
    left=3cm,
    right=1cm,
    top=2cm,
    bottom=2cm
]{geometry}

% Make titles not to have numbering
\newenvironment*{dummyenv}{}{}

\newcommand{\mysection}[1]{
    \addcontentsline{toc}{section}{#1}
    \begin{dummyenv}
        \bfseries\large #1
    \end{dummyenv}
}

\makeatletter
\patchcmd{\l@section}
  {\hfil}
  {\leaders\hbox{\normalfont$\m@th\mkern \@dotsep mu\hbox{.}\mkern \@dotsep mu$}\hfill}
  {}{}
\makeatother

% Here we go...
\title{БДЗ по прикладной криптографии}
\author{Фирсов Георгий, М21-507}

\begin{document}

\maketitle

\tableofcontents

\pagebreak
\mysection{Задание 1}

При известном заранее значении $D$ нарушитель может единожды найти такое значение $z$, что:
\begin{equation}
    \texttt{SHA256}(z) < \frac{2 ^ n}{D}.
\end{equation}

Это потребует некоторого времени, но идея в том, что это делается единожды и заранее.

Далее при обнародовании $x$ нарушитель вычисляет $y = z \oplus x$. При этом верна следующая
цепочка:
\begin{equation}
    H(x, y) = H(x, x \oplus z) = \texttt{SHA256}(x \oplus x \oplus z) = 
        \texttt{SHA256}(z) < \frac{2 ^ n}{D},
\end{equation}
то есть нарушитель может для каждого $x$ найти такой $y$, что $H(x, y) < \frac{2 ^ n}{D}$
за некоторое константное время.
\\

\mysection{Задание 2}

\pagebreak
\mysection{Задание 3}

\pagebreak
\mysection{Задание 4}

\pagebreak
\mysection{Задание 5}

\pagebreak
\mysection{Задание 6}

\pagebreak
\mysection{Задание 8}

\pagebreak
\mysection{Задание 9}

\end{document}
